%\documentclass{res}
%\usepackage{margin,line,pifont,palatino,fancyhdr}

\documentstyle[margin,line,pifont,palatino,fancyhdr]{res}

\addtolength{\headheight}{25pt}
\addtolength{\headsep}{15pt}

\oddsidemargin -0.5in
\evensidemargin -0.5in
\textwidth=6.0in
\itemsep=0in
\parsep=0in

\newenvironment{list1}{
  \begin{list}{\ding{113}}{%
      \setlength{\itemsep}{0in}
      \setlength{\parsep}{0in} \setlength{\parskip}{0in}
      \setlength{\topsep}{0in} \setlength{\partopsep}{0in}
      \setlength{\leftmargin}{0.17in}}}{\end{list}}

\newenvironment{list2}{
  \begin{list}{$\bullet$}{%
      \setlength{\itemsep}{0in}
      \setlength{\parsep}{0in} \setlength{\parskip}{0in}
      \setlength{\topsep}{0in} \setlength{\partopsep}{0in}
      \setlength{\leftmargin}{0.2in}}}{\end{list}}

\newenvironment{list3}[2]{
  \begin{list}{[\arabic{#1}]}{%
      \usecounter{#1}
      \setcounter{#1}{#2}
      \setlength{\itemsep}{0in}
      \setlength{\parsep}{0in} \setlength{\parskip}{0in}
      \setlength{\topsep}{0.04in} \setlength{\partopsep}{0in}
      \setlength{\leftmargin}{0.17in}}}{\end{list}}

%\fancypagestyle{respage}{%
%\fancyhf{}{Vita: VN Venkatakrishnan}
%\fancyhead[L]
%\fancyhead[R]{Page \thepage}
%}
%\pagestyle{respage}


%\pagestyle{fancy}
%\lhead{Vita: V.N. Venkatakrishnan}
%\chead{}
%\rhead{Page \thepage}
%\fancyfoot[c]{}

\pagestyle{fancy}
\lhead{{Curriculum Vitae:~Gennaro~Parlato }}
\chead{}
\rhead{Page \thepage}
\fancyfoot[c]{}
\renewcommand{\headrulewidth}{0.8pt}

\begin{document}
\thispagestyle{empty}
\name{
{\Large \bfseries \scshape Gennaro~~Parlato \hspace{5.5cm} Curriculum
Vitae~~~~{\small (\today)}
}}
%{\Large \bfseries \scshape Gennaro~~Parlato \hspace{7.7cm} Curriculum Vitae}}

\begin{resume}
%\vfill



%\includegraphics{imgGP.png}

\section{\bfseries \scshape Contact Information}

Computer Science Division \hfill \emph{Work:} +39 0874 404 176\\ 
Department of Bioscience and Territory \hfill \hfill \emph{Italy mobile:}   +39 338 90 100 62\\ %\emph{Fax:} +44 (0) 23 8059 3045\\
University of Molise \\ %\hfill \emph{mobile:} +39 338 90 100 62\\ 
Contrada Fonte Lappone \\% \hfill \emph{Italy mobile:}   +39 338 90 100 62\\
86090 - Pesche (IS), Italy\\

%\emph{E-mail:} ~~{\tt gennaro.parlato@unimol.it} \\
%\emph{Web:}  ~~{\bf\tt https://dibt.unimol.it/staff/parlato/} 
\emph{E-mail:} ~~{\tt gennaro.parlato@unimol.it} \\
\emph{Web:}  ~~{\bf\tt http://gennaro-parlato.github.io/} 





%\vfill
\section{\bfseries \scshape  Personal}
 \emph{Name:} {~~~~~~~~~~~~~ Gennaro}\\
 \emph{Last Name:} {~~~~~ Parlato}\\
 %\emph{Date of birth:} {~~ 14 March 1973}\\
 %\emph{Place of birth:} {\,\, Nocera Inferiore (SA), Italy}\\
 \emph{Citizenship:} {~~~\, British and Italian}\\
 %\emph{Tax Code:} {~~~~~~~~ PRLGNR73C14F912E  (Italy)}\\
 %\emph{Home Address:}{ Via Colle Impergola 2$/$A, Isernia, 86170, Italy}\\


%\vfill

\section{\bfseries \scshape  Education}

\begin{tabular}{@{}p{4.8in}p{1.8in}}
\emph{\bf Doctor of Philosophy (Ph.D.) in Computer Science} &
 \emph{11~~April~~2006}\\
     Universit\`a degli Studi di Salerno, Italy.&~\\
 Advisor: Prof.~Salvatore~La~Torre &~\\
 Thesis title: \emph{On the Model-Checking of Hierarchical and Recursive State Machines}&~\\
% Thesis submitted in November, 2001.&~\\
\end{tabular}

\begin{tabular}{@{}p{4.6in}p{1.8in}}
\emph{\bf Laurea Degree (M.Sc. equivalent) in Computer Science} & \emph{\,\,\,\,\,\,\,21~~March~~2002}\\
 Universit\`a degli Studi di Salerno, Italy.&~\\
Thesis title: \emph{Raggiungibilit\'a e Scoperta dei Cicli in Macchine a Stati Gerarchiche (in Italian)}&~\\
Advisor: Prof.~Margherita~Napoli &~\\
Grade: 110/110 summa cum laude &~\\
\end{tabular}




%\vfill

\section{\bfseries \scshape  Academic Positions}

%\begin{tabular}{@{}p{4.6in}p{1.8in}}
%{\bf Associate Professor, University of Molise } & \emph{~~~2018--Present}  \\
%Division of Physics, Computer Science and Mathematics,\\
%University of Molise, Italy.~\\
%§\end{tabular}

\begin{tabular}{@{}p{4.6in}p{1.8in}}
{\bf Associate Professor, University of Molise } & \emph{~~~Oct 2018--present}  \\
Computer Science division,\\
Department of Biosciences and Territory, \\
University of Molise, Italy.~\\
\end{tabular}

\begin{tabular}{@{}p{4.6in}p{1.8in}}
{\bf Associate Professor, University of Southampton } & \emph{~~~ Oct 2016--Sept 2018}  \\
Cyber Physical Systems Research Group,\\
School of Electronics and Computer Science, \\
University of Southampton, UK.~\\
\end{tabular}


\begin{tabular}{@{}p{4.6in}p{1.8in}}
{\bf Lecturer B in Automated Verification, University of Southampton } & \emph{~~~May 2011--Sept 2016}  \\
%(CUN: equivalent to the position of Associate Professor in Italy)\\
Electronic \& Software Systems Research Group,\\
School of Electronics and Computer Science, \\
University of Southampton, UK.~\\
\end{tabular}


\begin{tabular}{@{}p{4.6in}p{1.8in}}
{\bf Postdoctoral Researcher, LIAFA, CNRS, University of Paris 7 } & \emph{~~~Apr 2010--Apr 2011}  \\
Laboratoire d'Informatique Algorithmique: Fondements et Applications &~\\
(with Prof. Ahmed Bouajjani) ~\\
Paris, France.~\\
\end{tabular}



\begin{tabular}{@{}p{4.6in}p{1.8in}}
{\bf Postdoctoral Researcher, University of
Illinois at Urbana-Champaign} & \emph{~~~May 2006--Mar 2010}  \\
Dept. of Computer Science &~\\
(with Prof. P. Madhusudan) ~\\
Urbana, USA.~\\
\end{tabular}



\begin{tabular}{@{}p{4.6in}p{1.8in}}
{\bf Postdoctoral  Researcher, Universit\`a degli Studi di Salerno} & \emph{~~~Jan 2007--Dec 2009}  \\
Dept. of Computer Science &~\\
(with Prof. Margherita Napoli) ~\\
Fisciano, Italy.~\\
\end{tabular}

\begin{tabular}{@{}p{4.6in}p{1.8in}}
{\bf Postdoctoral  Researcher, Universit\`a degli Studi di Salerno} & \emph{~~~Feb 2006--Aug 2006}  \\
Dept. of Computer Science &~\\
(with Prof. Margherita Napoli)~\\
Fisciano, Italy.~\\
\end{tabular}





%\vfill

\section{\bfseries \scshape  Areas of Research}
\begin{itemize}

\item Program analysis, Testing and Verification
\item Model Checking 
\item Analysis of Concurrent Software
\item Learning
\item Logics, and Automata Theory 
\item Computer Security and Access Control
\end{itemize}
%My research interests are in the areas of program analysis and verification, verification of systems with infinitely many states, software model-checking, logics, automata theory, and graph theory.

%\begin{itemize}
% \item \emph{Software analysis and verification:} ~~software analysis, automatic verification, heap-analysis
% \item \emph{Model-checking:} ~~algorithms, temporal logics, tools
% \item \emph{Theory:} ~~logic, automata theory, graph theory.
%\end{itemize}

%\vfill


\section{\bfseries \scshape  Current Research Projects}
\begin{itemize}
   \item Model checking abstractions of concurrent programs using fixed-point ({\sc GetAFix})
   \item Decidable automata with auxiliary storage
   \item Sequentialization for the analysis of concurrent programs ({\sc CSeq})
   \item Verification of programs running under weak memory models
   \item Decidable logics for programs with data structures and data (STRAND)
   \item Verification of access control models ({\sc Vac})
 % \item Language-based security for distributed ledger 
   \item Automatic verification of heap-manipulating programs ({\sc Blob})
   \item  Programs Analysis in the Clouds  ({\sc PAC})

\end{itemize}




%\vfill


\section{\bfseries \scshape  Software/Tools Developed}
\begin{itemize}
   \item {\sc Cseq} -- Sequentialization Tools for Concurrent C Programs\\
         {\tt https://github.com/CSeq/Overview}
   \item {\sc GetAFix} --  A Symbolic Model-checker for Recursive Programs.\\
         {\tt http://madhu.cs.illinois.edu/getafix/}   
%   \item {\sc Grinder} -- A tool for verifying C programs using graph-representation of computations.\\
%         {\tt http://users.ecs.soton.ac.uk/gp4/cav13.html}
   \item {\sc Strand} -- Decidable logics/SMT solver for reasoning with heaps.\\
         {\tt https://engineering.purdue.edu/$\sim$xqiu/strand/index.html}
   \item {\sc Vac} -- Verifier of Access Control\\
         {\tt http://users.ecs.soton.ac.uk/gp4/VAC.html}
   \item {\sc PAC} -- Program Analysis in the Clouds\\
         %{\tt https://github.com/CSeq/Overview}
   
\end{itemize}


%\vfill


\section{\bfseries \scshape  Honors and Awards}
\begin{itemize}
   \item Spring 2021 Amazon Research Award!
   \item  Silver medal at ``Competition on Software Verification'' (SV-COMP'22), concurrency category\\\
    Tool:  Lazy-CSeq
   \item  Gold medal at ``Competition on Software Verification'' (SV-COMP'21), concurrency category\\\
    Tool:  Lazy-CSeq
   \item  Gold medal at ``Competition on Software Verification'' (SV-COMP'20), concurrency category\\\
    Tool:  Lazy-CSeq
   \item  Silver medal at ``Competition on Software Verification'' (SV-COMP'18), concurrency category\\\
    Tool:  Lazy-CSeq

   \item  Silver medal at ``Competition on Software Verification'' (SV-COMP'17), concurrency category\\\
    Tool:  Lazy-CSeq-Abs

   \item  Bronze medal at ``Competition on Software Verification''  (SV-COMP'17), concurrency category\\\
    Tool:  Lazy-CSeq-Swarm 


   \item  Gold medal at ``Competition on Software Verification''  (SV-COMP'16), concurrency category\\\
    Tool: MU-CSeq-0.4

   \item Silver medal at ``Competition on Software Verification''  (SV-COMP'16), concurrency category\\
    Tool: Lazy-CSeq-1.0

   \item Gold medal at ``Competition on Software Verification''  (SV-COMP'15), concurrency category\\
    Tool: Lazy-CSeq
   \item Silver medal at ``Competition on Software Verification''  (SV-COMP'15), concurrency category\\
    Tool: MU-CSeq

   \item Gold medal at ``Competition on Software Verification''  (SV-COMP'14), concurrency category\\
    Tool: Lazy-CSeq
   \item Silver medal at ``Competition on Software Verification''  (SV-COMP'14), concurrency category\\
    Tool: MU-CSeq

   \item Silver medal at ``Competition on Software Verification''  (SV-COMP'13), concurrency category\\
    Tool: CSeq

   \item Fellow of The Higher Education Academy (UKSPF Descriptor 2) [March 2014]
%   \item ``Annual Adventures in Research'' Grant, University of Southampton. [May 2012]
%   \item ``7k scholarship project'' Grant, University of Southampton. [February 2012]
   \item Teaching qualification in France for ñMaitre de Conferencesî in Computer Science [February 2011]

\end{itemize}



%\newpage
\bibliographystyle{plain}
\bibliography{ok,more}



%\vfill

\section{\bfseries \scshape  Invited \\Lectures}

%{\bf Program Committee:}\\
%\begin{list1}
%   \item[]
%   {\em FORMATS and FTRTFT} ~\\
%   Joint Conference on Formal Modelling and Analysis of Timed Systems
%   (FORMATS) and \\
%   Formal Techniques in Real-Time and Fault Tolerant Systems(FTRTFT),\\
%   Grenoble, France, 2004.
%\end{list1}





{\bf C2C-trans as a Design Methodology for Software Verification Tools}\\
The Workshop on
Democratizing Software Verification, 
(colocated with CAV 2022)\\
August 11, 2022, The Technion, Haifa, Israel





{\bf Parallel Bug-finding in Concurrent Programs Via Reduced Interleaving Instances}\\
Guest talk @ Amazon,\\
9 Sept 2020, streaming on Amazon Chime


{\bf Finding Rare Concurrent Programming Bugs}\\
Workshop on Verification of Distributed Systems (VDS)\\
19-21 June 2018, Marrakesh, Morocco


{\bf Finding Rare Concurrent Programming Bugs:
An Automatic, Symbolic, Randomized, and Parallelizable Approach}\\
15th International Colloquium on Theoretical Aspects of Computing (ICTAC),\\
12-19 October 2018, Stellenbosch, South Africa


{\bf A Pragmatic Bug-finding Approach for Concurrent Programs}\\
Institut de Recherche en Informatique Fondamentale (IRIF),\\
Paris, France, November 24, 2016


{\bf Security Analysis of Self-Administrated Role-Based Access Control through Program Verification}\\
Information Security Group, Department of Computer Science, University College London (UCL),\\
London, UK, July 2, 2015


{\bf On BMC Sequentializations of Concurrent Programs}\\
Computer Science Department, University of Oxford,\\
Oxford, UK, June 18, 2015

{\bf On Sequentializing Concurrent Programs}\\
UPMARC 7th Summer School on Multicore Computing,   \\
Uppsala University, Sweden, June 8-10, 2015 \\
{\tt http://www.it.uu.se/research/upmarc/events/SS2015/ss2015/Start.html}


{\bf Bounded Model Checking of Multi-Threaded C Programs via Lazy Sequentialization}\\
Laboratoire d'Informatique Algorithmique: Fondements et Applications  (LIAFA), \\
Paris, France, May 2014


{\bf The Tree-width of Decidable Problems}\\
FSTTCS post-conference workshop on ``Verification of Infinite-State Systems'',
Hyderabad, India, December, 2012 

{\bf Sequentializing Concurrent Programs}\\
Computer Science Department, University of Oxford,\\
Oxford, UK, January 17, 2012
{\tt http://www.cs.ox.ac.uk/seminars/699.html}


{\bf Decidable  Logics Combining Heap Structures and Data}\\
\begin{itemize}
\item 
LORIA-INRIA-Lorraine (CASSIS team)\\
Nancy, France, February 2011

\item
Workshop on Automata and Logic for Data Manipulating Programs,\\
Paris, France, December 2010

\item
ANR Veridyc Project,\\
Laboratoire d'Informatique Algorithmique: Fondements et Applications  (LIAFA), \\
Paris, France, October 2010



\end{itemize}


{\bf The Tree-Width of the Auxiliary Storage}\\
\begin{itemize}

\item 
School of Electronic Engineering and Computer Science at Queen Mary, University of London,
London, UK, March 2013

\item
IFIP WG 2.3: Working Group on Programming Methodology, Meeting 52\\ 
Winchester, UK. Set. 2011

\item 
\'Ecole normale sup\'erieure de Cachan (ENS Cachan)\\
Cachan, France, May 2010

\item Laboratoire Bordelais de Recherche en Informatique (LaBRI),\\ 
Bordeaux, France, May 2010
\item
Laboratoire d'Informatique Algorithmique: Fondements et Applications  (LIAFA), \\
Paris, France, April 2010

\end{itemize}

{\bf Writing Model-Checkers for Boolean Recursive Programs using a
Fixed-Point Calculus}\\
Laboratoire d'Informatique Algorithmique: Fondements et Applications  (LIAFA), \\
Paris, France, September 2009

%\vfill
\section{\bfseries \scshape  Grants}



\begin{tabular}{@{}p{4.6in}p{1.8in}}
{\bf Amazon Reseach Awards (ARA): Program Analysis in the Clouds (PAC): a distributed symbolic algorithm to scale up bug-finding in concurrent programs. } & \emph{~~~Aug 2021--Present}  \\
Total funding: 60,000 USD (Principal Investigator)\\
\end{tabular}



\begin{tabular}{@{}p{4.6in}p{1.8in}}
{\bf EPSRC Grant (EP/M008991/1): CONSEQUENCER: Sequentialization-based Verification of Concurrent Programs with FIFO Channels. } & \emph{~~~Mar 2015--Aug 2016}  \\
Total funding: \pounds ~120,000 (Principal Investigator)\\
\end{tabular}


\begin{tabular}{@{}p{4.6in}p{1.8in}}
{\bf EPSRC Grant (EP/P022413/1): VAC+: Verifier of Access Control. } & \emph{~~~Aug 2017--Present}  \\
Total funding: \pounds ~121,000 (Co-Investigator)\\
\end{tabular}


\begin{tabular}{@{}p{4.6in}p{1.8in}}
{\bf  GCHQ Grant: Language-based security for smart contracts in distributed ledger. } & \emph{~~~Dec 2015--Mar 2016}  \\
Total funding: \pounds ~21,000 (Co-PI)\\
\end{tabular}





\begin{tabular}{@{}p{4.6in}p{1.8in}}
{\bf ``Annual Adventures in Research'' Grant, University of Southampton} & \emph{~~~May 2015}  \\
Total funding: \pounds ~7,000 \\
\end{tabular}

\begin{tabular}{@{}p{4.6in}p{1.8in}}
{\bf ``Annual Adventures in Research'' Grant, University of Southampton} & \emph{~~~May 2012}  \\
Total funding: \pounds ~7,000 \\
\end{tabular}



\begin{tabular}{@{}p{4.6in}p{1.8in}}
{\bf ``7k scholarship project'' Grant, University of Southampton} & \emph{~~~Feb 2012}  \\
Total funding: \pounds  21,000 \\
\end{tabular}



%\vfill

\section{\bfseries \scshape  Graduate Students Pursuing Ph.D.}


\begin{tabular}{@{}p{4.6in}p{1.8in}}
{\bf Dr Enrico Steffinlongo } & \emph{Oct 2016-Feb-2018}  \\
Visiting PhD student from the University of Venice Ca' Foscari, Italy, Advisor Prof. Michele Bugliesi\\
(Now at {\bf Diff Blue}, Oxford, UK)  
\end{tabular}


\begin{tabular}{@{}p{4.6in}p{1.8in}}
{\bf Dr Mikhail Ramalho } & \emph{Feb 2015-Dec 2018}  \\
(co-advised with Denis A. Nicole) 
\end{tabular}

\begin{tabular}{@{}p{4.6in}p{1.8in}}
{\bf Dr Ermenegildo Tomasco } & \emph{May 2012--Mar 2018}  \\
Thesis title: {\em Separating Computation from Communication:
A Design Approach for
Concurrent Bug Finding}.\\
(Now official at {\bf Italian Income Revenue Authority}, Rome, Italy.) 
\end{tabular}

\begin{tabular}{@{}p{4.6in}p{1.8in}}
{\bf Dr Truc Lam Nguyen } & \emph{Oct 2013-May 2017}  \\
Thesis title: {\em A Pragmatic Verification Approach for Concurrent Programs}.\\
(Sr Software Data Engineer at {\bf Jagex}, Cambridge, UK.) 
\end{tabular}

\begin{tabular}{@{}p{4.6in}p{1.8in}}
{\bf Dr Omar Inverso } & \emph{Oct 2011-Oct 2015}  \\
Thesis title: {\em Bounded Model Checking of Multi-threaded Programs via Sequentialization}.\\
(Assistant Professor at {\bf Gran Sasso Science Institute}, Aquila, Italy.) 
\end{tabular}




\section{\bfseries \scshape  Postdocs}

\begin{tabular}{@{}p{4.6in}p{1.8in}}
{\bf Giulio Garbi } & \emph{Apr 2022-present}  \\
\end{tabular}

\begin{tabular}{@{}p{4.6in}p{1.8in}}
{\bf Mikhail Ramalho } & \emph{May 2016-Aug 2016}  \\
\end{tabular}

\begin{tabular}{@{}p{4.6in}p{1.8in}}
{\bf Truc Lam Nguyen } & \emph{Mar 2016-Aug 2016}  \\
\end{tabular}

\begin{tabular}{@{}p{4.6in}p{1.8in}}
{\bf Omar Inverso } & \emph{Apr 2015-Nov 2015}  \\
\end{tabular}



%\vfill

%\newpage
%\vspace{2cm}

\section{\bfseries \scshape  Professional activities }

%{\bf Program Committee:}\\
%\begin{list1}
%   \item[]
%   {\em FORMATS and FTRTFT} ~\\
%   Joint Conference on Formal Modelling and Analysis of Timed Systems
%   (FORMATS) and \\
%   Formal Techniques in Real-Time and Fault Tolerant Systems(FTRTFT),\\
%   Grenoble, France, 2004.
%\end{list1}

{\bf Refereed journals and conferences:}\\
   Formal Methods in System Design, Springer\\
   Information and Computation, Elsevier Press\\
   International Journal of Foundations of Computer Science (IJFCS)\\
   Journal of Computer Security (JCS)\\
   Journal of Systems and Software (JSS)\\
   Journal on Software Tools for Technology Transfer (STTT)\\
   Theoretical Computer Science (TCS)\\
   IEEE Transactions on Software Engineering (TSE)
\vspace{-0.2cm}

%   Acta Informatica, Springer-Verlag.\\
   
   Int'l Conference on Automated Software Engineering (ASE)\\
   Int'l Conference on Computer Aided Verification (CAV)\\
   Int'l Conference on Concurrency Theory (CONCUR)\\
   Int'l Conference on Developments in Language Theory (DLT)\\
   Int'l Conference on Computer Aided Verification (FCT)\\
   Int'l Conference on Formal Methods in Computer-Aided Design (FMCAD)\\
   Int'l Workshop on Formal Methods for Industrial Critical Systems (FMICS)\\
   Int'l Foundations of Software Science and Computation Structures (FOSSACS)\\
   Int'l Conference on Foundations of Software Technology and Theoretical Computer Science (FSTTCS)\\
   Int'l Symposium on Games, Automata, Logics and Formal Verification (GandALF)\\
   Int'l Colloquium on Automata, Languages and Programming (ICALP)\\
   Int'l Conference on Formal Engineering Methods (ICFEM)\\
   Int'l Conference on Software Engineering (ICSE)\\
   Int'l Symposium on Logic in Computer Science (LICS)\\
   Int'l Symposium on Mathematical Foundations of Computer Science (MFCS)\\
   Int'l Symposium on Principles of Programming Languages (POPL)\\
   Int'l SPIN Workshop on Model Checking of Software (SPIN)\\   
   Int'l Symposium on Theoretical Aspects of Computer Science (STACS)\\
   Int'l Competition on Software Verification (SV-COMP)\\
   Int'l  Conference on Tools and Algorithms for the Construction and Analysis of Systems (TACAS)\\
   Int'l Symposium on Trustworthy Global Computing (TGC)\\
   Int'l Conference on Verification, Model Checking, and Abstract Interpretation (VMCAI)\\
\vspace{-0.3cm}

{\bf Service}\\
\vspace{-0.19cm}
\begin{list1}

\item
PC Member: {\bf \bf VSTTE 2022}, Trento, Italy.\\
14$^{th}$ Working Conference on Verified Software: Theories, Tools, and Experiments.\\  

\item
PC Member: {\bf OVERLAY 2021}, Padova, Italy.\\
3$^{rd}$  Workshop on Artificial Intelligence and fOrmal VERification, Logic, Automata, and sYnthesis.

\item
PC Member: {\bf LATA 2020 \& 2021}, Milan, Italy.\\
14$^{th}$-15$^{th}$  Int'l Conference on Language and Automata Theory and Applications.

\item
PC Member: {\bf GandALF 2020}, Brussels, Belgium.\\
11$^{th}$ Int'l Symposium on Games, Automata, Logics and Formal Verification.

\item
PC Member: {\bf GandALF 2019}, Bordeaux, France.\\
10$^{th}$ Int'l Symposium on Games, Automata, Logics and Formal Verification.


\item
PC Member: {\bf \bf VSTTE 2018}, Oxford, UK.\\
10$^{th}$ Working Conference on Verified Software: Theories, Tools, and Experiments,\\ 
part of FLoC 2018. 

\item
PC Member:  {\bf LATA 2018},   Ramat Gan, Israel.\\
12$^{th}$ Int'l Conference on Language and Automata Theory and Applications.

\item
PC Member: {\bf GandALF 2018}, Saarbr\"ucken, Germany.\\
9$^{th}$ Int'l Symposium on Games, Automata, Logics and Formal Verification.

\item
PC \& Jury Member: {\bf SV-COMP 2017}, Uppsala, Sweden.\\
6$^{th}$ Int'l Competition on Software Verification.

\item
ERC Member: {\bf CAV 2016}, Toronto, Ontario, Canada.\\
     28$^{th}$ Int'l Conference on Computer Aided Verification.

\item

PC \& Jury Member: {\bf SV-COMP 2016}, Eindhoven, Netherlands.\\
5$^{th}$ Int'l Competition on Software Verification.

\item

PC Member:  {\bf ICSE 2016},   Austin, TX, USA.\\
38$^{th}$ Int'l Conference on Software Engineering.


\item

PC Member: {\bf TGC 2015}, Madrid, Spain.\\
10$^{th}$ Int'l Symposium on Trustworthy Global Computing.


\item

PC Member: {\bf FCT 2015},  Gdansk, Poland.\\
20$^{th}$ Int'l Symposium on Fundamentals of Computation Theory.

\item

PC Member: {\bf SPIN 2015}, Stellenbosch, South Africa.\\
22$^{nd}$ Int'l SPIN Workshop on Model Checking of Software.


\item

PC \& Jury Member: {\bf SV-COMP 2015}, London, UK.\\
4$^{th}$ Int'l Competition on Software Verification.

\item
PC Member: {\bf GandALF 2015}, Genoa, Italy.\\
5$^{th}$ Int'l Symposium on Games, Automata, Logics and Formal Verification.


\item

PC \& Jury Member: {\bf SV-COMP 2014}, Grenoble, France.\\
3$^{rd}$ Int'l Competition on Software Verification.


\item

PC Member: {\bf CAV 2012}, Berkeley, USA.\\
24$^{th}$ Int'l Conference on Computer Aided Verification.


\item
PC Member: {\bf MFCS 2012}, Bratislava, Slovakia.\\
37$^{th}$ Int'l Symposium on Mathematical Foundations of Computer Science.

\item
PC Member: {\bf GandALF 2010}, Minori, Italy.\\
1$^{st}$ Int'l Symposium on Games, Automata, Logics and Formal Verification.

\end{list1}


%\vfill


\section{\bfseries \scshape  Teaching \\Experience}

%Teaching assistant in Theory of Computation,
%Automata, and Formal Languages classes at Universit\`a degli Studi di Salerno, Italy.
%Spring Semester 2005 (60 hours).

%{\bf Fall }: Theory of Computing (COMP2011)\\
{\bf Semester 1, AY 2021/22, U. Molise}: Networking Security and Software Security\\
{\bf Semester 1, AY 2019/20, AY 2020/21, AY 2021/22, U. Molise}: Program Analysis\\
{\bf Semester 1, AY 2019/20, AY 2020/21, AY 2021/22, U. Molise}: Algorithms and Data Structures\\
{\bf Semester 2, AY 2018/19, U. Molise}: Algorithms and Data Structures\\
{\bf Semester 1, AY 2018/19, U. Southampton}: Theory of Computing (COMP2210)\\
{\bf July 2018}:  10-hour module on Program Verification, PhD program\\
~~~~~(IMT School for Advanced Studies Lucca, Italy)\\
{\bf May 2018}:  2018 Spring School on Theoretical Computer Science: Software Verification\\ 
(6-hour module), Aussois, France\\
{\bf Semester 1, AY 2017/18, U. Southampton}: Theory of Computing (COMP2210)\\
{\bf Semester 2, AY 2017/18, U. Southampton}: Algorithmics (COMP1201)\\
{\bf Semester 1, AY 2016/17, U. Southampton}: Theory of Computing (COMP2210)\\
{\bf Semester 1, AY 2017/17}: Topics in Computer Science (COMP6233)\\
{\bf Semester 2, AY 2016/17, U. Southampton}: Algorithmics (COMP1201)\\
{\bf April 2016}: 20-hour module on Program Verification, PhD program\\
~~~~~(IMT School for Advanced Studies Lucca, Italy)\\
{\bf Semester 1, AY 2015/16, U. Southampton}: Theory of Computing (COMP2210)\\
{\bf Semester 1, AY 2015/16}: Topics in Computer Science (COMP6233)\\
{\bf June 2015}: Lecture series at UPMARC Summer School on Multicore Computing\\ 
(Uppsala University, Sweden)\\
{\bf Semester 2, AY 2014/15}: Automated Software Verification (COMP6210)\\
{\bf Semester 1, AY 2014/15}: Theory of Computing (COMP2210)\\
{\bf Semester 2, AY 2013/14}: Formal Design of Systems (COMP6004)\\
{\bf Semester 1, AY 2013/14}: Theory of Computing (COMP2210)\\
{\bf Semester 2, AY 2012/13}: Formal Design of Systems (COMP6004)\\
{\bf Semester 1, AY 2012/13}: Theory of Computing (COMP2011)\\
{\bf Semester 2, AY 2011/12}: Formal Design of Systems (COMP6004)\\



%\vfill

\section{\bfseries \scshape  Research Experience}


\begin{tabular}{@{}p{4.6in}p{1.8in}}
{\bf Visiting Reseacher, IMT School for Advanced Studies Lucca}&
 \emph{~~~July-August 2018}\\
Computer Science and System Engineering research group&~\\
Hosted by Rocco De Nicola and Mirco Tribastone.~\\
Lucca, Italy~\\
\end{tabular}


\begin{tabular}{@{}p{4.6in}p{1.8in}}
{\bf Visiting Reseacher, University of Salerno}&
 \emph{~~~July-August 2017}\\
Department of Computer Science &~\\
Hosted by Salvatore La Torre~\\
Fisciano, Italy~\\
\end{tabular}


\begin{tabular}{@{}p{4.6in}p{1.8in}}
{\bf Visiting Researcher, IMT School for Advanced Studies Lucca}&
 \emph{~~~Apr 2016}\\
Computer Science and System Engineering research group&~\\
Hosted by Rocco De Nicola and Mirco Tribastone.~\\
Lucca, Italy~\\
\end{tabular}


\begin{tabular}{@{}p{4.6in}p{1.8in}}
{\bf Visiting Researcher, University of Salerno}&
 \emph{~~~Apr 2016}\\
Department of Computer Science &~\\
Hosted by Salvatore La Torre~\\
Fisciano, Italy~\\
\end{tabular}


\begin{tabular}{@{}p{4.6in}p{1.8in}}
{\bf Visiting Researcher, Stellenbosh University}&
 \emph{~~~Mar 2016}\\
Department of Computer Science &~\\
Hosted by Bernd Fischer~\\
Stellenbosh, South Africa~\\
\end{tabular}



\begin{tabular}{@{}p{4.6in}p{1.8in}}
{\bf Visiting Researcher, LIAFA, CNRS, University of Paris 7}&
 \emph{~~~May 2014}\\
Laboratoire d'Informatique Algorithmique: Fondements et Applications &~\\
Hosted by Ahmed Bouajjani ~\\
Paris, France~\\
\end{tabular}


\begin{tabular}{@{}p{4.6in}p{1.8in}}
{\bf Visiting Reseacher, LIAFA, CNRS, University of Paris 7}&
 \emph{~~~March 2012}\\
Laboratoire d'Informatique Algorithmique: Fondements et Applications &~\\
Hosted by Ahmed Bouajjani ~\\
Paris, France~\\
\end{tabular}



\begin{tabular}{@{}p{4.6in}p{1.8in}}
{\bf Visiting Reseacher, LIAFA, CNRS, University of Paris 7}&
 \emph{~~~October 2011}\\
Laboratoire d'Informatique Algorithmique: Fondements et Applications &~\\
Hosted by Ahmed Bouajjani ~\\
Paris, France~\\
\end{tabular}


\begin{tabular}{@{}p{4.6in}p{1.8in}}
{\bf Visiting Reseacher, University of
Illinois at Urbana-Champaign}&
 \emph{~~~July 2011}\\
 Dept. of Computer Science, \\
Hosted by P. Madhusudan ~\\
Urbana, USA~\\
\end{tabular}


\begin{tabular}{@{}p{4.6in}p{1.8in}}
{\bf Visiting Reseacher, University of
Illinois at Urbana-Champaign}&
 \emph{~~~Aug 2010}\\
 Dept. of Computer Science, \\
Hosted by P. Madhusudan ~\\
Urbana, USA~\\
\end{tabular}



%\vfill
\section{\bfseries \scshape  Lectures, Colloquia, and Conference Presentations}

{\bf  Reasoning about Data Trees using CHCs}\\
34th International Conference on Computer Aided Verification (CAV),
The Technion, Haifa, Israel, August 2022

{\bf  Concurrent Program Verification with Lazy Sequentialization and Interval Analysis}\\
International Conference on Networked Systems (NETYS),
Marrakech, Marocco, May 2017


{\bf  Verifying Concurrent Programs by Memory Unwinding}\\
International Conference on Tools and Algorithms for the Construction and Analysis of Systems (TACAS),
London, UK, April 2015

{\bf On the Path-Width of Integer Linear Programming}\\
International Symposium on Games, Automata, Logics and Formal Verification (GandALF)\\
Verona, Italy, September 2014


{\bf Bounded Model Checking of Multi-Threaded C Programs via Lazy Sequentialization}\\
International Conference on Computer Aided Verification (CAV)\\
Vienna, Austria, July 2014


{\bf Scope-bounded Multistack Pushdown Systems:  Fixed-Point, Sequentialization, and Tree-Width}\\
Int'l Conference on Foundations of Software Technology and Theoretical Computer Science (FSTTCS)\\
Hyderabad, India, December 2012

{\bf On Sequentializing Concurrent Programs}\\
International Static Analysis Symposium (SAS)\\
Venice, Italy, September 2011

{\bf Getting Rid of Store-Buffers in the Analysis of Weak Memory Models}\\
International Conference on Computer Aided Verification (CAV)\\
Cliff Lodge, Snowbird, Utah, USA, July 2011


{\bf The Tree-Width of the Auxiliary Storage}\\
Symposium on Principles of Programming Languages (POPL)\\
Austin, TX, USA, January 2011


{\bf The Tree-Width of the Auxiliary Storage}\\
Formal Method seminar,
University of Illinois at Urbana-Champaign\\
Urbana, IL, USA, March 2010


{\bf Analyzing Recursive Programs using Fixed-point Calculus}\\
Formal Method seminar,
University of Illinois at Urbana-Champaign\\
Urbana, IL, USA, October 2009

{\bf Reducing Context-bounded Concurrent Reachability to Sequential Reachability}\\
Conference on Computer Aided Verification (CAV)\\
Grenoble, France, June 2009


{\bf An Infinite Automaton Characterization of Double Exponential Time}\\
Conference on Computer Science Logic (CSL)\\
Bertinoro, Italy, September  2008

{\bf Context-Bounded Analysis of Concurrent Queue Systems} \\
International Conference on Tools and Algorithms for the Construction and Analysis of Systems (TACAS)\\ Budapest, Hungary, April 2008

{\bf On the Complexity of LTL Model-Checking of Recursive State Machines}\\
International Colloquium on Automata, Languages and Programming (ICALP)\\
Wroclaw, Poland, July 2007

{\bf Hierarchical and Recursive State Machines with Context-Dependent Properties}\\
International Colloquium on Automata, Languages and Programming (ICALP)\\ Eindhoven, The Netherlands, July 2003


%\vfill
%\section{\bfseries \scshape  Member of Projects}



%{\bf VERIDYC ANR-09-SEGI-016, 2009-2012}. France.\\
%Partners: EDF, LIAFA/Modelisation et Verification, CEA Saclay, LSV/INFINI,
%VERIMAG/DCS Team.
%(https://sites.google.com/site/veridyc/)

%{\bf NSF Grant: Formal Security Analysis of Access Control Models and
%Extensions. NSF Program: Trustworthy Computing.} USA. \\
%PIs: Vijay Atluri, Rutgers University.\\
%P. Madhusudan, University of Illinois at Urbana-Champaign.


%{\bf FARB projects, 2002-2009}. University of Salerno, Italy.











%\newpage

%\section{\bfseries \scshape  References}


%References available upon request.



%\begin{tabular}{@{}p{3in}p{3in}}



%{\bf Prof. Ahmed~Bouajjani}    & {\bf Prof. P.~Madhusudan}  \\
%(Postdoc advisor)    & (Postdoc advisor)  \\
%LIAFA, University Diderot (Paris 7)  & University of Illinois at Urbana-Champaign \\
%Dept. of Computer Science    &  Dept. of Computer Science \\
%Case 7014,                   & 3226 Siebel Center, \\
%75205 Paris cedex 13,  & 201 N. Goodwin Avenue, \\
%FRANCE.                & Urbana, Illinois 61801-2302, U.S.A. \\
%E-mail: abou@liafa.univ-paris-diderot.fr  ~~ & E-mail: madhu@uiuc.edu \\
%Tel: ~+33 1 57 27 92 64; ~~Fax: ~+33 1 57 27 94 09    & Tel: ~+1-217-244-1323; ~~Fax: ~+1-217-265-4035 \\
%{\tt http://www.liafa.jussieu.fr/}$\sim${\tt abou/} & {\tt http://www.cs.uiuc.edu/}$\sim\!\!\!${\tt ~madhu/}\\


%~ & ~\\
%~ & ~\\
%~ & ~\\

%{\bf Prof. Vladimiro Sassone}&\\
% (current employer/co-worker)&   \\
% University of Southampton & \\
% Dept. of Electronics and Computer Science&  \\
%Highfield, Southampton SO17 1BJ, UK &  \\
%E-mail: vs@ecs.soton.ac.uk  & \\
%Tel: ~~+44 2380 599009; ~~Fax: +44 2380 593045 & \\
%{\tt http://www.ecs.soton.ac.uk/people/vs} &\\
%\end{tabular}

















\end{resume}
\end{document}
